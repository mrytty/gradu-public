\chapter{Summary}
\label{chap:conclusions}

The purpose of this thesis was to give an overview of arbitrage-free pricing methodology and affine short-rate processes used in interest rate modeling and credit risk. These kind of models have a long history starting from the late 1970's. Their heyday was pre-financial crisis of 2007-08. After the crisis, many significant changes have occurred in interest rate markets, for example:
\begin{itemize}
	\item IBOR rates are not anymore considered as riskless and relevant spreads have widened.
	\item Credit adjustments are required for unsecured positions. 
	\item Multi-curve pricing has been the industry standard at least for linear products such as swaps.
	\item Extraordinary monetary policies of central banks have caused negative interest rates, which were earlier considered impossible. As such, the possibility of negative rates in Va\v{s}\'{i}\v{c}ek--model was considered a flaw earlier.
\end{itemize}
Although short-rate models have been surpassed by market models in practice, it is an interest question consider how short-rate models fare in the current market structures.

Short-rate models have several theoretical short-comings. The first is conceptual, there is no actual instantaneous short-rate. It is purely theoretical concept created to explain how the interest rates are formed. Although having an unobservable process as a main ingredient of a theory is troubling, it can be forgiven if implied theoretical structure is otherwise logically sound and it can produce accurate results. The viability of Black-Scholes option pricing methodology is based on the assumption that the future volatility of the stock process can be accurately inferred, even if it is not actually observable. Same can be true for short-rate modeling of interest rates. If the observed term structure is coherent with the implications of a hypothesized short-rate process, then the model might be useful even if the model might be fundamentally wrong.

The second main theoretical short-coming of short-rate modeling is the fact that it is mainly concerned of a single point, the next infinitesimal future time-step. As such, it is not fat-fetched to hypothesize that short-rate models have hard time to explain complex temporal evolution of interest rate curve. For example, we have demonstrated that for affine short-rate models with one factor, the long-rates are perfectly correlated. This is a severe limitation but it can be mitigated with the introduction of multiple factors, time-varying parameters or  technique of dynamic extension. 

However, short-rate models are not without merits. They are conceptually simple to understand. Affine models are analytically tractable with explicit analytical bond pricing formulas. Some models even have explicit analytical bond option pricing formulas which can be converted to the price caps and floors.

The calibration of models to the market data was inconclusive in the sense that we did not achieve consistent results. For simple models without credit risk, although there were some precise calibrations, no model was consistently accurate. Since we employed stochastic optimization algorithms for calibration, there is no certainty that global minimums were found. The curse of dimensionality makes the optimization problem very hard for multi-factor models. Thus a bad fit does not indicate that the model is unsuitable for the observed data. Since single-factor models have a manageable number of parameters, we could expect optimization to be fairly dependable. For $A(0,1)+$--model, the alternative calibration replicated the original parameter values almost perfectly. For $A(0,1)+$--model, the alternative calibration produced significantly different parameter values but the accuracy was very similar. Therefore we can infer with reasonable confidence that $A(0,1)+$ and $A(1,1)+$--models can not necessarily fit post-crisis term-structures. Since some of the considered multi-factor models of family $A(M,N)+$ offered decent accuracy, we believe that two or three factor models could be used to fit the recent observed data. As the alternative calibrations led to significantly different parameters and calibration errors, the inference about model quality of multi-factor models of type $A(M,N)+$ is not reliable.

Since the calibration data included maturities ranging from overnight rate to 30-year rate, the observed rate structure is complex. Calibration to subset of these maturity ranges will likely produce significantly better fits. This reasoning is supported results of the dynamic calibration of Euribor rates ranging from one week to annual maturity, which generally give significantly better accuracy. 

For simple models with credit risk, the calibration errors were large. However, calibration was only attempted with recent data and only models with two-factors were considered. As we had seen, single-factor models did not perform well in this environment for interest rate curves and these factors had to explain both the interest rate and spread curves. It is probably that models with more factors could work better.

The calibration methodology employed had severe short-comings. Although differential evolution has the desired ability to explore the optimization space, it probably wasted lots of function calls to explore infeasible regions. On the other, since the alternative optimizations tended not to converge original points, it seems that meta-parameters for the optimization were misspecified. Also it is not clear if differential evolution is the best choice for this kind of calibration. Particle swamp optimization or simulated annealing might have been better alternatives. The large number of solutions near the optimization borders suggest that those borders might have been inappropriate. On the other hand. Since the shifts were stopped at borders, the borders were also likely to be hit during optimization tries.

It could be interesting to test how well these affine models and their dynamic extensions compare against more modern models such as SABR when they are calibrated to recent volatility structures. Since descendants of LIBOR models are based on the assumption of log-normal distribution, negative interest rates causes problems that require unnatural solutions such as shifts or normality assumptions that may lower model quality. Negative rates are possible for affine models with Gaussian factors, but this probably does not compensate for inferior volatility fabrics of these models.

