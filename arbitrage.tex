\chapter{An introduction to arbitrage pricing theory}
\label{chap:arbitrage}

We shall now review the basic setting and results of arbitrage pricing theory.

\section{Discrete one period model}

In order to gain an insight into the basic concepts in mathematical finance, an overview of discrete one period model is given in this chapter. This treatment is based on \textcite[pp. 5--34]{bjork2004arbitrage} and \textcite[pp. 3--12]{duffie2010dynamic}. \added[comment={Added contribution}]{While the context follows these sources closely, the presentation should be unique. For example, we do not assume the ecistance of a risk-free rate.}. A more complete overview of discrete model with multiple time periods can be found in \textcite[pp. 33--85]{musielarutkowski2005martingale}.

Let $(\Omega, \Pf)$ be a discrete probability space with $\Omega = \{ \omega_1, \omega_2, \ldots , \omega_m \}$ and we assume that $\Pf (\omega ) > 0$ for all $\omega \in \Omega$. The discrete one period market model is the sample space $\Omega$ coupled with the price process $S$. We assume that at the time $t=0$ the price vector $S_0 \in \R^n$ is a known constant and at the time $t=1$ it is a random vector $S_1 : \Omega \rightarrow \R^n$. Thus our model consists of one time step with $n$ stochastic assets. 

A portfolio $w$ is a vector in $\R^n$. Since we do not put any restrictions on $w$, we assume that there are no restriction on buying or short selling of assets in this market. We denote the starting value of portfolio $w$ by $V_0^w = w^{\top} S_0$ and the final value of it is a random variable $V_1^w = w^{\top} S_1$. 

We say that the portfolio $w$ is an arbitrage portfolio if either
  \begin{enumerate}[labelindent=\parindent, leftmargin=*]
    \item $V_0^w < 0$ and $V_1^w \geq 0$ or
    \item $V_0^w \leq 0$ and $V_1^w > 0$
  \end{enumerate}
$\Pf$-surely. We say that the market with starting prices is arbitrage free if there are no arbitrage portfolios. We denote
  \begin{align}
    \label{marketmatrixindiscretesetting}
    M = \begin{bmatrix} S_1 ( \omega_1 ) & S_1 ( \omega_2 ) & \ldots & S_1  ( \omega_m ) \end{bmatrix}_{n \times m} = \begin{bmatrix} M_1^{\top} \\ M_2^{\top} \\ \vdots \\ M_n^{\top} \end{bmatrix}_{n \times m}
  \end{align}
where $M_i^{\top} \in \R_m$ is the vector for terminal prices of $i$th asset. Now the absence of arbitrage is equivalent to that neither
  \begin{enumerate}[labelindent=\parindent, leftmargin=*]
    \item $w^{\top} S_0 = V_0^w < 0$ and $w^{\top} M \geq 0$ nor
    \item $w^{\top} S_0 = V_0^w \leq 0$ and $w^{\top} M > 0$
  \end{enumerate}
does not hold for any portfolio $w$. We recall that for vectors, we use $x > 0$ to denote that every component of $x$ is positive. Likewise $x \geq 0$ denotes component-wise non-negativity.

\subsection{The first fundamental theorem of asset pricing in discrete one period model}

A state-price vector is a vector $x \in \R^m$ satisfying $S_0 = Mx$ and $x > 0$. This definition can be understood in the following context. If $x \in \R^m$ is a state-price vector and $x = \beta y$, where $\left| y \right| = 1 $ and $\beta = \left| x \right|$, then $S_0 = Mx = \beta M y$. where $My$ is the expected value of $S_1$ under the probability measure where the state probabilities are given by the vector $y$. The probability vector of the original probability space $\Pf$ does not have to be a state-price vector.

It may happen that the state-price vector does not exists, but if it does, then $V_0^{w} = w^{\top} S_0 = w^{\top} Mx$ for all portfolios $w$ and this implies that arbitrage does not exist. Suppose that $w \in \R^n$ and $x$ is a state-price vector. Now $x > 0$ and $w^{\top} S_0 = w^{\top} Mx < 0$ implies that $w^{\top} M$ has a negative component. Likewise $x > 0$ and $w^{\top} S_0 = w^{\top} Mx = 0$ implies that not all elements of $w^{\top} M$ are positive.

We shall now prove the reverse implication using a variant of hyperplane separation theorem, which can be found in many standard textbooks for convex optimization.

\begin{thm}[Hyperplane separation theorem]
Let $A$ and $B$ be closed convex subsets of $R^s$. If either of them is compact, then there exists $0 \not = x \in R^s$ such that
  \begin{align}
    a^{\top} x < b^{\top} x
  \end{align}
for all $a \in A$ and $b \in B$.
\end{thm}

\begin{lemma}
\label{discreteonetimefirstfundamentallemma}
The market $(\Omega, S)$ is arbitrage free if and only if a state-price vector exists.
\end{lemma}

\begin{proof}
\added[comment={Added reference}]{The argument presented here is essentially the same as the one found in} \textcite[p. 4]{duffie2010dynamic}. We denote
  \begin{align}
    A = \{ \ (-w^{\top} S_0, w^{\top} M ) \in \R^{m+1} \ | \ w \in \R^n \ \}
  \end{align}
and $C = \R_+^{m+1}$. Now the absence of arbitrage is equivalent to $A \cap C = \{ 0 \}$. We need to only prove that the absence of arbitrage implies that the state-price vector does exist.

It is clear that $A$ is a closed and convex linear subspace of $\R^{m+1}$. We consider the function $f$ defined by $z \mapsto z / \left| z \right|$ for all $0 \not = z \in \R^{m+1}$. Since $C$ is a convex subset, it is easy to to see that the convex closure of $B = f(C \setminus \{ 0 \})$ is a convex and compact subset of $C$. 

By the hyperplane separation theorem, there exists $0 \not = y \in R^{m+1}$ such that $a^{\top} y < b^{\top} y$ for all $a \in A$ and $b \in B$. Since $0 \in A$, $0 < b^{\top} y$ for all $b \in B$, which implies that $0 < c^{\top} y$ for all $0 \not = c \in C$. Coordinate vectors are in $C$ and this means that $y > 0$. Since $A$ is a subspace, $a \in A$ implies that $-a \in A$ and this means that $0 \leq a^{\top} y < \left| c^{\top} y \right|$ for all $a \in A$ and $0 \not = c \in C$. Thus $a^{\top} y = 0 $ for all $a \in A$.

If $y = (y_1, y_2, \ldots , y_{m+1})^{\top} \in \R^{m+1}$ and $y^* = (y_2, \ldots , y_{m+1})^{\top} \in \R^{m}$, then
  \begin{align}
    y_1 w^{\top} S_0 = w^{\top} M y^*
  \end{align}
for all $w \in \R^n$, which implies $x = y^* / y_1 > 0$ is a state-price vector.
\end{proof}

If the market is arbitrage free and $x = (x_1, \ldots x_m)^{\top}$ is a state-price vector, then by denoting 
  \begin{align}
    \beta = \sum_{i=1}^m x_i
  \end{align}
and $\Pm (\omega_i) = q_i = x_i / \beta > 0$ for all $i = 1,2, \ldots ,m$, we have a new probability measure $\Pm$ on $\Omega$. This measure has the property
  \begin{align}
    \label{martingalemeasureequationdiscrete}
    S_0 = M x = \beta M \frac{x}{\beta} = \beta \E_{\Pm} \left( S_1 \right) .
  \end{align}
Since we assumed that $\Pf (\omega) > 0$ for all $\omega \in \Omega$, it is easy to see that $\Pm (\omega) > 0$ for all $\omega \in \Omega$.
  
A deflator $d$ is a strictly positive process, so that $d_0 > 0$ and $d_1(\omega) > 0$ for all $\omega \in \Omega$. Now we may define relative price processes
\begin{align}
S^d_0 &= \frac{S_0}{d_0}, \\
S^d_1 &= \frac{S_1}{d_1}
\end{align}
with regards to the deflator $d$.

\begin{lemma}
	\label{discreteonetimemartingalemeasurelemma}
	The market has a state-space vector if and only if the market with relative prices has a state-price vector.
\end{lemma}

\begin{proof}
	Let
	\begin{align}
	M^d = \left[ \ \frac{S_1 ( \omega_1 )}{ d_1 (\omega_1) } \ \frac{S_1 ( \omega_2 )}{ d_1 (\omega_2) } \ \ldots \ \frac{S_1  ( \omega_m )}{ d_1 (\omega_m) } \ \right]_{n \times m} .
	\end{align}
	If $x = (x_1, x_2, \ldots , x_m)^{\top} > 0$ and
	\begin{align}
	x^d = d_0^{-1} ( x_1 d_1 (\omega_1), x_2 d_1 (\omega_2), \ldots , x_m d_1 (\omega_m) )^{\top},
	\end{align}
	then $d_0 M^d* x^d = Mx$. Therefore $S_0 = Mx$ if and only if $S^d_0 = M^d x^d$, where $x^d \in \R^m$ and $x^d > 0$.
\end{proof}  
  
Inspired by these observations, we define that a measure $\Pm$ that satisfies 
  \begin{enumerate}[labelindent=\parindent, leftmargin=*]
    \item $\Pf( \omega ) = 0$ if and only if $\Pm ( \omega ) = 0$ (meaning that the measures share null sets) and
    \item there exists a deflator $d$ such that 
    \begin{align}
    		S^d_0 = \E_{\Pm} \left( S^d_1 \right)
   	 \end{align}
  \end{enumerate}
is an equivalent martingale measure induced by deflator $d$. By Lemma \ref{discreteonetimemartingalemeasurelemma}, the original market is arbitrage-free if and only if the deflated market is arbitrage-free, which is equivalent to
    \begin{align}
		S^d_0 = \beta \E_{\Pm} \left( S^d_1 \right)
	\end{align}
for some $\beta \in \R_+$ and measure $\Pm$. We can define a new deflator $e$ by $e_0 = d_0$ and $e_1 = \beta d_1$. Now
    \begin{align}
		S^e_0 = \beta \E_{\Pm} \left( S^e_1 \right)
	\end{align}
meaning that $\Pm$ is an EMM induced by $e$. Thus we see that the market is arbitrage-free if and only if it has an EMM.

The curious aspect of the probabilities of the EMM is that they are not influenced by the original probability measure $\Pf$ beyond the fact that they share the same sets of non-zero probabilities. In fact, the probabilities of the EMM is defined by the original price vector $S_0$ and the state space $M$.

Assume that $\Pm$ is an equivalent measure. Since $\Pf$ is probability measure with no null sets, we may define a new random variable called Radon-Nikod\'{y}m derivative as
\begin{align}
\Lambda ( \omega ) = \frac{ \Pm ( \omega ) }{ \Pf ( \omega ) } 
\end{align}
for all $\omega \in \Omega$. If $X$ is a random variable, then
\begin{align}
\E_{\Pf} (\Lambda X) = \sum_{i=1}^m \Pf (\omega_i) \frac{ \Pm ( \omega_i ) }{ \Pf ( \omega_i ) } X( \omega_i ) = \E_{\Pm} (X) 
\end{align}
so we see that the expectations under different measures are linked by the Radon-Nikod\'{y}m derivative. By this we see that $X$ is a martingale under the measure $\Pm$ if and only if $\Lambda X$ is a martingale under the measure $\Pf$.

A num\'{e}raire is any asset with only positive prices. If one of the assets is a num\'{e}raire with initial price $s_0$ and terminal price $s_1( \omega )$, then we use it as a deflator and define relative price process
  \begin{align}
    S_0^* &= \frac{S_0}{s_0}, \\
    S_1^* &= \frac{S_1}{s_1}.
  \end{align}
If the market is arbitrage-free, then by Lemma \ref{discreteonetimemartingalemeasurelemma} there exists such $\beta \in \R_+$ and measure $\Pm$ that 
\begin{align}
S^*_0 = \beta \E_{\Pm} \left( S^*_1 \right) .
\end{align}
But since $s$ is one of the assets, it must hold that $\beta = 1$ meaning that the measure $\Pm$ is an equivalent martingale measure.

We can now state the first fundamental theorem of asset pricing that weaves these different concepts together.

\begin{thm}[First fundamental theorem of asset pricing for discrete one period model]
The following are equivalent in a discrete one period market model: 
  \begin{enumerate}[labelindent=\parindent, leftmargin=*]
    \item the market is arbitrage free,
    \item a state-price vector exists and
    \item an equivalent martingale measure $\Pm$ exists.
  \end{enumerate}
If $\Pm_d$ is an EMM induced by a deflator $d$ and $\Pm_e$ is an EMM induced by a deflator $e$, then
  	\begin{align}
		S_0 = d_0 \E_{\Pm_d} \left( S^d_1 \right) = e_0 \E_{\Pm_d} \left( S^e_1 \right) .
	\end{align}	
In particularly, if the market has a num\'{e}raire $s$ and there is no arbitrage, then a martingale measure induced by $s$ satisfies
  	\begin{align}
		S_0 = s_0 \E_{\Pm} \left( \frac{S_1}{s_1} \right).
	\end{align}
\end{thm}

A risk-free asset is a num\'{e}raire with a constant terminal price $z$. We assume that the risk-free asset has initial price of $1$ and terminal price of $1+r$, where $r > -1$. Thus if a risk-free asset exists and there is no arbitrage, an EMM $\Pm$ induced by a risk-free asset satisfies
	\begin{align}
		S_0 = \E_{\Pm} \left( \frac{S_1}{1+r} \right) .
	\end{align}


\subsection{The second fundamental theorem of asset pricing in discrete one period model}

\added[comment={Added reference}]{The arguments presented in this section is} essentially the same as the ones in \textcite[pp. 31--34]{bjork2004arbitrage}

How should we then price derivatives in this model? A contingent claim $X$ is a random variable $X : \Omega \rightarrow \R$. If the the original market is arbitrage-free, then the first fundamental theorem of asset pricing implies that an equivalent martingale measure $\Pm$ induced by a deflator $d$ satisfies
  \begin{align}
    S_0 & = d_0 \E_{\Pm} \left( S^d_1 \right) .
  \end{align}
It would be natural to define
  	\begin{align}
		X_0 & = d_0 \E_{\Pm} \left( X^d_1 \right)
	\end{align}
to be the initial price of the claim $X$. But since there may be different EMMs, $X_0$ may not be well-defined. Thus it is vital to pose the question how many measures there are. If the market is arbitrage free, then the equation $S_0 = M x$ has a solution by Lemma \ref{discreteonetimefirstfundamentallemma}. By basic linear algebra, this solution is unique if and only if $\Kernel{ ( M ) } = 0$. We also know that null space is the orthogonal compliment of the row space meaning that
  \begin{align}
    \label{linearalgebraduality}
    \Kernel{ ( M ) } = \Image{ (M^{\top}) }^{\perp} 
  \end{align} 
and this suggests that we have a closer look at the image set $\{ x^{\top} M \ | \ x \in \R^n \} $. But the image set is just the set of all possible portfolios of the original market.

If there is such portfolio $w$ that $w^{\top} S_1 = X$ $\Pf$-surely, then we say that $X$ is replicated by the portfolio $w$. This is equivalent to
	\begin{align}
		X \in \{ x^{\top} M \ | \ x \in \R^n \} .
	\end{align}
By the first fundamental theorem of asset pricing, we see that if the contingent claim $X$ can be replicated and the market has no arbitrage, then for any pair of an EMM $\Pm$ and a replicating portfolio $w$, we have that 
  \begin{align}
    V_0^w = d_0 \E_{\Pm} \left( w^{\top} S^d_1 \right) = d_0 \E_{\Pm} \left( \frac{X}{d} \right),
  \end{align}
where $d$ is a deflator inducing $\Pm$. The left hand side depends only on the replicating portfolio $w$ and the right hand size depends only on the measure $\Pm$ and the deflator. This implies that $V_0^{w_1} = V_0^{w_2}$ for any portfolios $w_1, w_2$ which replicates $X$. 

We define $Z_0 = (w^{\top} S_0, S_0^{\top})^{\top}$, $Z_1 = (w^{\top} S_1, S_1^{\top})^{\top}$. If $\Pm$ is an EMM with a deflator $d$, then
  	\begin{align}
		Z_0 = d_0 \E_{\Pm} \left( Z^d_1 \right)
	\end{align}
if and only if 
  	\begin{align}
		w^{\top} S_0 &= d_0 \E_{\Pm} \left( w^{\top} S^d_1 \right) ,\\
		S_0 &= d_0 \E_{\Pm} \left( S^d_1 \right) .
	\end{align}
Hence, if the market is arbitrage-free, then the arbitrage-free price of a replicated contingent claim $X$ is
$w^{\top} S_0$, where $w$ is any replicating portfolio.

Now we define that the market is complete if every contingent claim can be replicated. This means that if $M$ is defined as in \ref{marketmatrixindiscretesetting}, then the marker is complete if and only if 
  \begin{align}
    \Image{ (M^{\top}) } = \{ M^{\top} w \ | \ w \in \R^n \} = \{ (w^{\top} M)^{\top} \ | \ w \in \R^n \} = \R^m
  \end{align} 
meaning that the matrix $M$ has a rank of $m$. By duality in \ref{linearalgebraduality}, we see that this is equivalent to $\Kernel{ ( M ) } = 0$. Therefore the market is complete if and only if the solution to the equation in Lemma \ref{discreteonetimefirstfundamentallemma} has a unique solution. Thus we have proved the second fundamental theorem of asset pricing.

\begin{thm}[Second fundamental theorem of asset pricing]
Assume that the market is arbitrage free. The market is complete if and only if there is a deflator that induces a unique equivalent martingale measure. Then every EMM induced by a given deflator is unique. If the market has a num\'{e}raire, then the market is complete if and only if EMM induced by the num\'{e}raire is unique.
\end{thm}

Completeness also implies that the market has $m$ linearly independent asset price processes since the matrix $M$ has a rank of $m$. This means, in a sense, that for every risk dimension is covered be tradable assets and therefore every possible contingent claim can be replicated.

So we have identified three distinct scenarios which are from best to worst:
  \begin{enumerate}[labelindent=\parindent, leftmargin=*]
    \item The market model is arbitrage free and complete which is equivalent to the fact the there exists a deflator with a unique martingale measure. Every contingent claim can be given a unique price which is the cost of replicating portfolio. Or, if $d$ is deflator with induced EMM $\Pm$, then the initial arbitrage-free price of a contingent claim $X$ is
    	\begin{align}
    		\label{deflatorinducedpricingformula}
	    	X_0 = d_0 \E_{\Pm} \left( \frac{X_1}{d_1} \right) .
    	\end{align}
    This price does not depend on the choice of the deflator $d$ as long as the deflator induces an EMM.
    \item The market is arbitrage free but not complete. Then every deflator which induces an equivalent martingale measure have several EMMs. Every replicated contingent claim can be given a unique arbitrage-free price which is the cost of replicating portfolio. This price is given by Equation \ref{deflatorinducedpricingformula}. Contingent claims that could not be replicated may not be priced in the sense of the Equation \ref{deflatorinducedpricingformula}.
    \item The market has arbitrage which makes pricing rather meaningless. 
  \end{enumerate}

Thus, if the market does not allow arbitrage, then we may price every replicated contingent claim can be given an arbitrage-free price. This price is the cost of the cost of replicating portfolio $w$ which is equal to
    \begin{align}
		V^w_0 = d_0 \E_{\Pm} \left( \frac{V^w_1}{d_1} \right) ,
	\end{align}
where $d$ is any deflator that induces an EMM $\Pm$. If the market has a num\'{e}raire, then we may use this as the deflator.
  


