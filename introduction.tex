\chapter{Introduction}

\section{Overview of the arbitrage-free asset pricing}

\begin{figure}[H]
	\centering
	\includegraphics[width=1\linewidth]{pic/bis}
	\caption{OTC derivatives notional amount outstanding. Source: \cite{bis2019statistics}}
	\label{fig:bis2019statistics}
\end{figure}

\begin{figure}[H]
	\centering
	\includegraphics[width=1\linewidth]{pic/bis2}
	\caption{OTC derivatives gross market values. Source: \cite{bis2019statistics}}
	\label{fig:bis2019statistics2}
\end{figure}

What is an appropriate price of an asset? Financial assets, which are contracts over other assets, are often homogeneous and standardized. Intuitively their prices should mainly depend on investors' preferences over the distributions of future returns. Investors have different preferences.  Risk appetite, regulatory requirements, investment horizons and business concerns differentiates them. Secondly, history has showed that predicting the future is difficult and, as a corollary, detailed prediction of price behavior should be also hard. Without auguries, the derivation of exact probabilities is impossible. 

According to \textcite{cochrane2009asset}, "asset pricing theory tries to understand the prices or values of claims to uncertain payments (p. xiii)". Asset pricing theory has two main approaches, absolute and relative pricing. Absolute pricing tries to model sources of economic risks and/or underlying preferences, and derive prices from these. The canonical examples are CAPM and other equilibrium models. In contrast, relative pricing does not try the model the whole investment universe. In relative prices, some asset prices are exogenous and other assets are priced relative to these. Black-Scholes--model (\cite{blackscholes1973pricing}) is canonical example of this approach. The demarcation of these approaches is not clear-cut. CAPM assumes the equilibrium prices as given and Black-Scholes--model makes a fundamental assumption about the distribution of asset returns. (\cite[pp. xiii--xiv]{cochrane2009asset})

The relative pricing approach is often called as arbitrage pricing methodology. The common theme in these models is that we make an assumptions about the underlying distribution of the exogenous prices. If the market is arbitrage-free, then these exogenous prices contain meaningful information about probabilities for different outcomes probability distribution. Thus we do not try to derive actual probability distribution, but we construct a new probability measure, which is commonly called as the equivalent martingale measure (EMM).  These implied distributions are then used to construct replicating portfolios. By a construction of a perfect hedge, preferences do not affect the price of a replicating portfolio. 

The seed of the arbitrage pricing theory is usually attributed to the thesis by \textcite{bachelier1900theorie}. But the research by \textcite{jovanoviclegall2001does} suggest that Bacheliers work was antecedent by another French \textcite{regnault1863calcul}. The insight of these early pioneers was not try to outguess the market but the idea was model the price movements by Brownian motion\footnote{The problem with Brownian motion is that with non-zero probability the price will be negative given enough time. A more valid approach is to use geometric Brownian motion which do not have this problem. But geometric Brownian motion is still continous so it will not model the jumps in the price process.} and use the distribution to price the derivative contract. Although the work of Bachelier was temporarily forgotten, it resurfaced in the 1950's and influenced a host of work \parencite{samuelson1973mathematics}. Thus Regnault and Bachelier can be also seen as the forefathers of the efficient-market hypothesis which was largely developed in modern sense by \textcite{fama1965behavior} and \textcite{samuelson1965proof}.

These early modern attempts to price options were not satisfactory even to the authors who derived them\footnote{See, for example, the lamentation in \textcite{samuelson1965rational}.} until the seminal work by \textcite{blackscholes1973pricing}. The major insight in this and subsequent work was that, in certain idealized frictionless market model, the cash flow of the option could be perfectly replicated by trading the underlying stock and a risk-free bond. Since there was no practical difference in executing this trading strategy and owning the option, there was a unique no-arbitrage price for the option. This price depends neither the risk aversion of the inverstor nor his views of the market (expect for the volatility of stock price).

The mathematical formulation in \textcite{blackscholes1973pricing} was lacking. According to \textcite[p. 129]{musielarutkowski2005martingale}, it was \textcite{bergman1982pricing} who first noted that the trading strategy used by \textcite{blackscholes1973pricing} was neither risk-free nor self-financing. The modern arbitrage asset pricing theory was formalized by \textcite{harrisonkreps1979martingales} in discrete time and \textcite{harrisonpliska1981martingales} in continuous time (see also \textcite{harrisonpliska1983stochastic}). Several authors have expanded these works and \textcite{delbaenschachermayer1998fundamental} (and references within) present one version of the theory in very general a semi-martingale setting. 

According to \textcite{bis2019statistics}, the outstanding amount of over the counter derivatives was over 598 trillion USD. However, since 

Figures \ref{fig:bis2019statistics} and \ref{fig:bis2019statistics2}

\subsection{Arbitrage-free interest rate models}

Interest is the time value of money. A vast literature exist to explain the mechanics driving the term-structure of interest rates. Short-rate modeling began under this framework and was based on macro-economical argumentation. According to \textcite[p. 161]{duffie2010dynamic}, the earliest example of markovian term-structure model is by \textcite{pye1966markov}. His approach influenced \textcite{merton1974pricing}, a paper which contains an example of gaussian short-rate model. The first relevant short-rate model is the famous Va\v{s}\'{i}\v{c}ek--model by \textcite{vasicek1977equilibrium}. Another widely used short-rate model is CIR--model by \textcite{coxingersollross1985theory}. Since the original formulation of these models are based on economic equilibrium argumentation, they are often called either equilibrium or fundamental models and risk-preferences have to be explicitly formulated and they influences prices. Although fundamental models may have economically sound argumentation, their basic problem is practical impossibility to fit them to the observed interest-rate structure.

These early models can be also cast in arbitrage pricing framework, but we lose the economic justification behind the interest rate process. Arbitrage-free short-rate modeling just assumes that the interest rates are generated by a given exogenous stochastic process, a short-rate. This rate is mathematically modeled by an It\'{o}-process. Observed rates are then stochastic integrals of this short rate. By this approach, pricing of interest rate derivatives leads to solving stochastic differential equations. When the short-rate model has constant parameters, then the observed interest-rate structure may not be matched. \textcite{ho1986term} introduced a short-rate model with time-varying parameter, which can be chosen to fit the observed term-structure. Va\v{s}\'{i}\v{c}ek--model was extended with time-varying parameters in  \textcite{hull1990pricing}.

Since short-rate models capture only a point, they have hard time capturing the complex dynamics of the term-structures. For example, we later show that in single-factor affine term-structure models rates of different maturities are perfectly correlated. An alternative to the short-rate approach is to directly model the entire term structure of interest rates. An early successor was HJM--framework  (introduced in \cite{HJM1990bondpricing} and \cite{HJM1992bondpricing}), which models the entire forward rate process. It should be noted Ho-Lee and Hull-White models can be cast as special cases of HJM model. A major problem for general HJM models is that these are not necessary markovian (\cite{ritchken1995volatility}).

Nowadays so called market models are widely used in pricing. According to \textcite[p. 182]{wu2009interest}, traders had assumed that LIBOR and swap rates were log-normal processes and used Black's formula\footnote{See \textcite{black1976pricing} and section \ref{blackformula}.} to quote volatility when pricing caps and swaptions since the early 1990's. Theoretical justification for these practices were finally found in 1997 by a series of articles by \cite{bracegatarek1997market}, \textcite{miltersen1997closed} and \textcite{jamshidian1997libor}. Market models use LIBOR (or other market interest rates) as the fundamental objects and assume that they can be modeled as log-normal processes. Thus they are often called as LIBOR models. The main feature of market models is that cap and swaption pricing is given by Black's formula and they can be made to fit given term-structure and volatility structures. Since the LIBOR rates follow log-normal rates, these models have to be extended to handly negative interest rate. This is often done by modeling a shifted log-normal process. A popular extension of market model is SABR volatility model by \textcite{hagan2002managing}.

The global financial crisis of 2007-08 has had a major impact on interest-rate modeling. One of the significant additions is the multi-curve framework (\textcite{mercurio2009interest}). Before the crisis, the spread between overnight indexed swap (OIS) and LIBOR curves were minimal and LIBOR curve was both the discounting and forward rate generating curve. During the crisis, this spread widened dramatically and afterwards it has been necessary the model these curves separately. 

\subsection{Intensity based modeling of credit risk}

An early example of credit risk modeling is \textcite{merton1974pricing}, which uses Black-Scholes--model to price corporate debt subject to credit risk. In Merton's model corporate assets are assumed to follow geometric brownian motion and the corporate debt consists of single zero-coupon bond. Now the equity can be seen as a call option on corporate assets with the strike price of the face value of a bond at the maturity. Thus the equity price is given by the Black-Scholes formula and the bond price is the difference of asset and equity values. \textcite{merton1974pricing} is extended by proprietary Moody's KMV model. 

An alternative to the structural credit models are dynamic models. One approach is to use so called intensity based modeling of default. In this framework, the default time is a stopping time with a intensity process\footnote{See chapter \ref{chap:intensity} and section \ref{sec:stoppingtime}.}. This intensity process may be influenced by properties of economy or the underlying entity. In the arbitrage-free settings intensity may be modeled by either Poisson or Cox processes. This approach was pioneered by \textcite{artzner1995default}, \textcite{jarrow1995pricing} and \textcite{lando1998cox}.

The popularity intensity based modeling is due to synergies with short-rate models of interest rates. When set-up correctly, the default intensity process is the credit spread process. Then the pricing of debt and related instruments can be made using the machinery developed for the short-rate processes. \textcite{schonbucher2001libor} has extended market models to cover credit intensity modeling.

After the global financial crisis of 2007-08, the credit risk modeling has gained importance. For example, Basel III framework requires that the prices of unsecured derivative positions has to be corrected with CVA\footnote{CVA is credit value adjustment and pricing has also account DVA, debit value adjustment.}, which accounts for the counterparty credit risk (\cite{basel2015cva}).

\section{Overview of the thesis}

The purpose of this thesis is to have a rough overview arbitrage-free pricing methodology and affine short-rate processes used in interest rate modeling and credit risk. Although short-rate models have been eclipsed by market models, they still have their uses in risk management, portfolio management and scenario planning.

As the short-rate models were developed before global financial crisis of 2007-08, we test how well they can be fitted to the post-crisis interest rate data. We also try to test calibrate a combined short-rate and credit spread model to post-crisis bond price data.

\subsubsection*{Chapter \ref{chap:instuments}}

Chapter \ref{chap:instuments} gives a basic overview of the common interest rates and financial instruments.

\subsubsection*{Chapter \ref{chap:arbitrage}}

Chapter \ref{chap:arbitrage} gives a very brief introduction to arbitrage pricing theory. We first develop discrete one period model in order to highlight the basic concepts of arbitrage pricing such as justification behind martingale measures and laws of asset pricing. This treatment is based on \textcite[pp. 5--34]{bjork2004arbitrage}. All the proofs are detailed as we feel that they give insight to martingale measures.  

After that the arbitrage pricing in continuous markets is overviewed. This treatment is not rigorous. Measure theoretical justifications and arguments are simply omitted although some basic results are  presented in Appendix \ref{chap:math}.

\added]{While not strictly necessary for the empirical work in this thesis, the arbitrage pricing theory is essential for understanding the peculiarities in interest rate and credit spread modeling.}

\subsubsection*{Chapter \ref{chap:shortrate}}

Chapter \ref{chap:shortrate} starts with the derivation of term-structure equation and the overview of the fundamental and preference-free models. We have a brief overview of the single-factor Va\v{s}\'{i}\v{c}ek and CIR--models. 

We also cover multi-factor affine $A(M,N)$--models, which are a combination of $M$ Va\v{s}\'{i}\v{c}ek and $N-M$ CIR--models. The Gaussian processes can be chosen to be correlated but square-root processes has to be uncorrelated. The theory of affine term-structure models was developed for example in \textcite{brown1994interest}, \textcite{duffie1994multi} and \textcite{duffiekan1996yield}. The main advantage of these affine models is that they have analytical bond prices which make calibration by term-structure easy. By dynamic extension shifting\footnote{See \textcite{brigomercurio2001deterministic} and section \ref{sec:dynamicextension}.}, we even could make sure that term-structures fit perfectly.
Many of these models also have analytical bond option prices, which allows us to price caps and floors and for single factor models, this allows also easy pricing of swaptions. Therefore they can be easily fit to volatility structures also. Even if the model does not have a bond option pricing formula, we may use Fourier transformations to have semi-analytical pricing of bond options\footnote{See \textcite{heston1993closed} and section \ref{chap:fourier}.}.

\subsubsection*{Chapter \ref{chap:intensity}}

Chapter \ref{chap:intensity} is an overview of the basic interest rates and financial instruments. For default modeling based on a Cox process\footnote{Cox process is also called as doubly stochastic process, since in addition to the stochastic stopping time, the time dependent intensity process is also stochastic.}. We show that in this setting the pricing of defaultable zero-coupon bonds can be made using the machinery of Chapter \ref{chap:shortrate}. We develop pricing formulas bonds and credit default swaps using different assumptions about the default.     

\subsubsection*{Chapter \ref{chap:empirical}}

Chapter \ref{chap:empirical} presents the methods and results from the empirical work done in this thesis. We test some of the methods develop in earlier chapters. In order to find suitable initial starting value for descending optimization algorithm, we tried to scan the problem space with a differential evolution (DE) algorithm. Due to computational limitations, the sample sizes were small compared to the dimension of the problem space. As such, the algorithm did not produce consistent results.

Simple affine models were fitted to various interest rate and yield curves with and without default risk. Fittings of these curves proved to be challenging for the models as the maturities ranged from overnight rates (or 6-month rates) to 30-year rates. No model provided satisfactory fitting in every case. Fitting done by models with credit risk component proved bad overall. However, due to the problems with optimization algorithm, we can not decisively rule that the models will provide bad fits with the used data.

Dynamic fitting of Euribor rates ranging from 1-week rate to 1-year rate was also attempted to single factor models. Depending on the time-period, the fits ranges from horrible to rather satisfactory. As the short end of the rate curve has been rather flat after 2014, calibrated curves had very little errors.

Due to data and computational limitations, fitting to the volatility structures was not attempted.  

The code with the Jupyter notebook used in analysis can be found at \url{https://github.com/mrytty/gradu-public} (\textcite{rytty}).

\subsubsection*{Appendices \ref{chap:math} and \ref{chap:charts}}

Appendix \ref{chap:math} is a brief review of mathematical machinery and methods needed in the earlier chapters.

Appendix \ref{chap:charts} presents some additional data charts and tables from the empirical work. 
