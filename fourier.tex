\section{Option valuation using Fourier inversion method}
\label{chap:fourier}

Now we show how to value options using the Fourier inversion method introduced by \textcite{heston1993closed}. Our approach is based on \textcite[pp. 222--233]{nawalkabeliaevasoto2007dynamic}.

We make the following assumptions. The interest rate model is affine $A_M(N)$ model. Thus
	\begin{align}
		r(t) = \delta(t) + \sum_{m=1}^M X_m(t) + \sum_{i=1}^{N-M} Y_i(t) .
	\end{align}
The correlated gaussian process are
\begin{align}
\dx Y_i(t) &= - k_i Y_i(t) \dx t + \nu_i \dx W_i (t),
\end{align}
where $W_i$ is a Wiener process and 
\begin{align}
\dx W_i (t) \dx W_j (t) &= \rho_{ij} \dx t    
\end{align}
for all $i,j = 1,2, \ldots, N-M$. Here $-1 < \rho_{ij} = \rho_{ji} < 1$ and $\rho_{ii} = 1$. The $M$ square-root processes are
\begin{align}
\dx X_m(t) = \alpha_m ( \theta_m - X_m(t) ) \dx t + \sigma_m \sqrt{ X_m(t) } \dx Z_m (t)
\end{align}
where $Z_m$ are independent Wiener process and
\begin{align}
\dx W_i (t) \dx Z_m (t) &= 0    
\end{align}
for all $i = 1,2, \ldots, N-M$ and $m = 1,2, \ldots, M$. 

We explicitly assume that $\delta(t) = 0$ for all $t \geq 0$ to keep notation simple and we do the dynamic extension of Section \ref{sec:dynamicextension} to add the $\delta(t)$ afterwards. Now zero-coupon bond prices are given by
	\begin{align}
		\label{fourierbondaffineassumption}
		\Bond(t,T) = \e^{A(\tau) - B^{\top} (\tau) X(t) - C^{\top} (\tau) Y(t) } ,
	\end{align} 
where $\tau = T-t$. This is a simplification from \textcite[p. 433--435]{nawalkabeliaevasoto2007dynamic}.
	
The price of a call option expiring on $S$ written on $T$-bond with strike price $K$ is
	\begin{align}
		c(t) &= \E_{\Pm_0} \left( \e^{ - \int_t^S r(s) \dx s } \left( \Bond(S,T)) - K \right)_+ \ | \ \F_t \right) \\
			&= \E_{\Pm_0} \left( \e^{ - \int_t^S r(s) \dx s } \left( \Bond(S,T)) - K \right) \1_{ \Bond(S,T) - K \geq 0 } \ | \ \F_t \right) \\
			&= \Bond(t,T) \Pi_1 - K \Bond(t,S) \Pi_2 ,
	\end{align}
where
	\begin{align}
		\Pi_1 &= \E_{\Pm_0} \left( \frac{\e^{ - \int_t^S r(s) \dx s } \Bond(S,T) \1_{ \Bond(S,T) \geq K }}{\Bond(t,T)} \ | \ \F_t \right) \\
		\Pi_2 &= \E_{\Pm_0} \left( \frac{\e^{ - \int_t^S r(s) \dx s } \1_{ \Bond(S,T) \geq K }}{\Bond(t,T)} \ | \ \F_t \right) .
	\end{align}
Here all the expectations are taken under the risk-free measure. We now write $\Pi_1$ under a different measure. As $\Bond(t,T) \geq K$ is equivalent to $\ln \Bond(t,T) \geq  \ln K$, we change the variable by $y = \ln \Bond(t,T)$ and get
	\begin{align}
		\label{fourierchangeofvariable}
		\Pi_1 &= \int_{\ln K}^{\infty} \left( \frac{\e^{ - \int_t^S r(s) \dx s } \Bond(S,T) }{\Bond(t,T)} f(y) \right) \ \dx y .
	\end{align}
We notice that 
	\begin{align}
		\xi_1 (t) &= \frac{\Bond(S,T) \Bank(t)}{\Bond(t,T) \Bank(S)} \\
		&= \frac{ \Bond(S,T) }{\Bond(t,T)} \e^{ - \int_t^S r(s) \dx s } .
\end{align}
is the Radon-Nikod\'{y}m derivative of $T$-forward measure with respect to risk-free measure. Thus
	\begin{align}
		\Pi_1 &= \int_{\ln K}^{\infty} \xi_1(t) f(y) \ \dx y \\
			&= \int_{\ln K}^{\infty} f_1(y) \ \dx y
\end{align}
where $f_1$ is the probability density function under the $T$-forward measure. Let $g_1$ be the characteristic function of $T$-forward measure, hence 
	\begin{align}
		g_1 (\omega) &= \int_{-\infty}^{\infty} \e^{i\omega y} f_1(y) \ \dx y \\
			&= \int_{-\infty}^{\infty} \left( \e^{i\omega y} \xi_1(t) f(y) \right) \ \dx y \\
			&= \E_{\Pm_0} \left( \e^{i\omega \ln \Bond(S,T)} \xi_1(t)  \ | \ \F_t \right) \text{ and } \\
		g_1 (\omega) \Bond(S,T) &= \E_{\Pm_0} \left( \e^{(1+i\omega) \ln \Bond(S,T)} \e^{ - \int_t^S r(s) \dx s } \ | \ \F_t \right)
	\end{align}
as, by Equation \ref{fourierbondaffineassumption},
	\begin{align}
		y = \ln \Bond(S,T) = A(\tau) - B^{\top} (\tau) X(t) - C^{\top} (\tau) Y(t) .
	\end{align}
Feynman-Kac theorem this expected value may be presented as a $N$-dimensional stochastic partial differential equation. \textcite{nawalkabeliaevasoto2007dynamic} shows that a solution is
\begin{align}
\exp \left( A_1^*(s) - \sum_{m=1}^M B_{1m}^*(s) X_m(t) - \sum_{i=1}^{N-M} C_{1m}^*(s) Y_i(t) \right) ,
\end{align}
where
\begin{align}
A_1^*(0) &= a_1 = A(U) ( 1+\boldsymbol{i} \omega ) \\ 
B_{1m}^*(0) &= b_{1m} = B_m(U) ( 1+\boldsymbol{i} \omega ) \\ 
C_{1i}^*(0) &= c_{1i} = C_i(U) ( 1+\boldsymbol{i} \omega )
\end{align}
for $i= 1,2, \ldots, N-M$ and $m=1,2, \ldots , M$. Here
\begin{align}
A_1^*(z) &= a_1  +  \frac{1}{2} \sum_{i=1}^{N-M} \sum_{j=1}^{N-M} \frac{\nu_i \nu_j \rho_{ij}}{k_ik_j} ( z - q_i C_i(z) - q_j C_j(z) \\ &+ q_iq_j \frac{1 - \exp^{ - (k_i+k_j)z } }{k_i+k_j} ) \\
&- 2 \sum_{m=1}^M \frac{\alpha_m \theta_m}{\sigma_m^2} \left( \beta_{3m} z + \log ( \frac{1 - \beta_ {4m} \exp^{ \beta_{1m} z } }{1 - \beta_{4m}} )  \right) \\
B_{1m}^*(z) &= \frac{2}{\sigma_m^2} \left( \frac{ \beta_{2m} \beta_{4m} \exp^{\beta_{1m} z} - \beta_{3m} }{ \beta_{4m} \exp^{\beta_{1m} z} - 1 } \right) \\
C_{1i}^* (z) &= \frac{1-q_i \exp^{-k_iz}}{k_i},
\end{align}
where
\begin{align}
q_i &= 1 - k c_{1j} \\
\beta_{1m} &= \sqrt{\alpha_m^2 + 2 \sigma_m^2} \\
\beta_{2m} &= \frac{- \alpha_m + \beta_{1m}}{2} \\
\beta_{3m} &= \frac{- \alpha_m - \beta_{1m}}{2} \\
\beta_{4m} &= \frac{ - \alpha_m - \beta_{1m} - b_{1m} \sigma_m^2 }{ - \alpha_m + \beta_{1m} - b_{1m} \sigma_m^2 }
\end{align}
for $i= 1,2, \ldots, N-M$ and $m=1,2, \ldots , M$. It should be noted that $a_1, b_{1m}$ and $c_{1i}$ are actually functions of $\omega$ and therefore $A_1^*, B_{1m}^*$ and $C_{1i}^*$ also depend on $\omega$.

Thus the characteristic function is
\begin{align}
g_1 (\omega) &= \frac{ \exp \left( A_1^*(s) - \sum_{m=1}^M B_{1m}^*(s) X_m(t) - \sum_{i=1}^{N-M} C_{1m}^*(s) Y_i(t)  \right) }{\Bond(t,T)} 
\end{align}
which allows us to calculate the values of characteristic functions. Now
	\begin{align}
		\Pi_1 &= \int_{\ln K}^{\infty} f_1(y) \ \dx y \\
			&= \frac{1}{2} + \frac{1}{\pi} \int_0^{\infty} \Re \left( \frac{\exp^{-\boldsymbol{i}\omega \log K} g_1(\omega)}{\boldsymbol{i} \omega} \right) \dx \omega
	\end{align}
which can be calculated numerically. \textcite{nawalkabeliaevasoto2007dynamic} note that this computation only requires that the model has analytical bond pricing formulas, so it can be utilized in variety of models.


We can solve $\Pi_2$ similarly. Instead of Equation \ref{fourierchangeofvariable}, we have
	\begin{align}
		\Pi_2 &= \int_{\ln K}^{\infty} \left( \frac{\e^{ - \int_t^S r(s) \dx s } \Bond(S,T) }{\Bond(t,T)} f(y) \right) \ \dx y .
	\end{align}
and similar reasoning shows that
	\begin{align}
		\Pi_2 = \frac{1}{2} + \frac{1}{\pi} \int_0^{\infty} \Re \left( \frac{\exp^{-\boldsymbol{i}\omega \log K} g_2(\omega)}{\boldsymbol{i} \omega} \right) \dx \omega .
	\end{align}
Now
	\begin{align}
		g_2 (\omega) &= \frac{ \exp \left( A_2^*(s) - \sum_{m=1}^M B_{2m}^*(s) X_m(t) - \sum_{i=1}^{N-M} C_{1m}^*(s) Y_i(t)  \right) }{\Bond(t,S)} 
	\end{align}	
where
\begin{align}
A_2^*(0) &= a_2 = A(U) ( \boldsymbol{i} \omega ) \\ 
B_{2m}^*(0) &= b_{2m} = B_m(U) ( \boldsymbol{i} \omega ) \\ 
C_{2i}^*(0) &= c_{2i} = C_i(U) ( \boldsymbol{i} \omega )
\end{align}
for $i= 1,2, \ldots, N-M$ and $m=1,2, \ldots , M$. Similarly as before 
\begin{align}
A_2^*(z) &= a_2  +  \frac{1}{2} \sum_{i=1}^{N-M} \sum_{j=1}^{N-M} \frac{\nu_i \nu_j \rho_{ij}}{k_ik_j} ( z - q_i C_i(z) - q_j C_j(z) \\ &+ q_iq_j \frac{1 - \exp^{ - (k_i+k_j)z } }{k_i+k_j} ) \\
&- 2 \sum_{m=1}^M \frac{\alpha_m \theta_m}{\sigma_m^2} \left( \beta_{3m} z + \log ( \frac{1 - \beta_ {4m} \exp^{ \beta_{1m} z } }{1 - \beta_{4m}} )  \right) \\
B_{2m}^*(z) &= \frac{2}{\sigma_m^2} \left( \frac{ \beta_{2m} \beta_{4m} \exp^{\beta_{1m} z} - \beta_{3m} }{ \beta_{4m} \exp^{\beta_{1m} z} - 1 } \right) \\
C_{2i}^* (z) &= \frac{1-q_i \exp^{-k_iz}}{k_i},
\end{align}
where
\begin{align}
q_i &= 1 - k c_{2j} \\
\beta_{1m} &= \sqrt{\alpha_m^2 + 2 \sigma_m^2} \\
\beta_{2m} &= \frac{- \alpha_m + \beta_{2m}}{2} \\
\beta_{3m} &= \frac{- \alpha_m - \beta_{2m}}{2} \\
\beta_{4m} &= \frac{ - \alpha_m - \beta_{2m} - b_{2m} \sigma_m^2 }{ - \alpha_m + \beta_{2m} - b_{2m} \sigma_m^2 }
\end{align}
for $i= 1,2, \ldots, N-M$ and $m=1,2, \ldots , M$. Again, it should be noted that $a_2, b_{2m}$ and $c_{2i}$ are actually functions of $\omega$ and therefore $A_2^*, B_{2m}^*$ and $C_{2i}^*$ also depend on $\omega$.
	
Since the computations above involves numerical integration, it is computationally costly. However \textcite{carrmadan1999optionvaluation} showed that Fast Fourier Transform (FFT) can utilized in the computation, which can reduce the complexity significantly. Sadly, this was not attempted in this thesis. 

As we explicitly assumed that $\delta = 0$, we can add the shift to the model by the extension method introduced in Section \ref{sec:dynamicextension}.


